\section{Compilazione ed Esecuzione}
\paragraph{Compilazione}
La compilazione viene gestita attraverso il toolkit \textit{Maven}, che ci permette di definire delle apposite routine\footnote{È definita anche una routine per \textit{pulire} attraverso il comando "\textit{mvn clean}"} che si occuperanno di effettuare il packing sia del client che del server di TURING in formato JAR.

\begin{lstlisting}[language=bash, caption="Compilazione tramite Maven"]
$ mvn package
...
[INFO] -------------------------------------------------------------------
[INFO] BUILD SUCCESS
[INFO] -------------------------------------------------------------------
[INFO] Total time:  14.655 s
[INFO] Finished at: 2019-xx-xxTxx:xx:xx+xx:xx
[INFO] -------------------------------------------------------------------
\end{lstlisting}
\pagebreak
A seguito della compilazione, dunque, verranno generati i seguenti file:
\begin{itemize}
	\item ./target/TURING-Client.jar
	\item ./target/TURING-Server.jar
\end{itemize}

\paragraph{Esecuzione}
Per eseguire i due JAR basta utilizzare il comando \textit{java -jar}, passando,come argomento, il file da eseguire.

\begin{lstlisting}[language=bash, caption="Esempio di esecuzione di TURING-Server.jar]
$ java -jar ./target/TURING-Server.jar -h
\end{lstlisting}

